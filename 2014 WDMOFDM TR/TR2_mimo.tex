\section{Optical MIMO Channel}\label{sec:mimo}
This section provides an introduction to an optical MIMO channel that can be established between multiple optical transmitting and receiving elements. Multiple transmitting elements can be established over dimensions such as space, time, frequency, wavelength, polarization, and others. Each receiving element must produce an output electrical signal that is proportional to the radiant flux incident on it from the dimension of interest. A typical optical MIMO channel can be modeled as a linear time invariant system and represented as
\begin{equation}
	\label{eqMIMOch}
	\vm{Y} = \vm{H}\vm{X} + \vm{W}
\end{equation}
$\vm{X}$ is an $N_{tx}$ dimensional vector representing the radiant flux emitted by each transmitting element. $\vm{Y}$ is an $N_{rx}$ dimensional vector representing the signal output from each receiving element. $\vm{H}$ is an $N_{rx}\times N_{tx}$ dimensional channel matrix where each element ($h_{ij}\in \vm{H}$) represents the conversion factor for signal output from the $i^{th}$ receiving element when radiant flux from $j^{th}$ transmitting element is incident on it after including the path-losses. $\vm{W}$ is an $N_{rx}$ dimensional noise vector typically modeled as additive white Gaussian noise (AWGN) independent of transmitted vector $\vm{X}$ i.e. $\vm{W}\sim\mathcal{N}({\bf{0}},\sigma_n^2\vm{I}_{N_{rx}})$ where $\sigma_n^2$ is noise variance and $\vm{I}_{N_{rx}}$ is an identity matrix of size $N_{rx}$.

The amount of radiant flux per solid angle emitted by a transmitting element in a certain angular direction $\phi$ is given by the angular radiant intensity distribution. An LED's radiant intensity distribution is usually characterized by a Lambertian distribution and is given by
\begin{equation}
	\label{eqLamb}
	L_m(\phi) = \twopartdef{{\frac{(m+1)}{2\pi}}cos^{m}(\phi)}{-\pi/2\leq\phi\leq\pi/2}{0}{\mbox{ else}}
\end{equation}
If $\Phi_{\frac{1}{2}}$ is the emission angle at which radiant intensity of a transmitting element is half its peak value (at $\phi=0^\circ$), the Lambertian order of that emission is $m=-ln(2)/ln(cos(\Phi_{\frac{1}{2}}))$.

A photodiode is a device that produces an electrical signal output that is proportional to the radiant flux incident on it. The photodiode effective area can be increased by using concentrator optics and this gain is given by
\begin{equation}
	\label{eqGain}
	G_{\eta}(\psi) = \twopartdef{\frac{\eta^2}{sin^2(\Psi_{\frac{1}{2}})}} {0\leq\psi\leq\Psi_{\frac{1}{2}}\leq\frac{\pi}{2}}{0}{\psi>\Psi_{\frac{1}{2}}}
\end{equation}
where $\psi$ is the angle of incidence of the flux, $\eta$ is the refractive index of the material that the optics are made of, and $\Psi_{\frac{1}{2}}$ is the half angle of the field of view of the concentrator.

Optical filters may be used at the receiver to acquire the wavelengths of interest while rejecting the ambient radiation. Depending on the type of filter used, its transmittance may be a function of the angle of incidence $\psi$. Let $T(\psi,\lambda)$ be the transmittance of an optical filter. If $S(\lambda)$ is the normalized SPD of incident radiation and $R(\lambda)$ is the responsivity of the receiving photodiode, the effective responsivity of the receiving element is given by
\begin{equation}
	\label{eqReff}
	R_e(\psi) = G_{\eta}(\psi)\int^{\lambda_{max}}_{\lambda_{min}}S(\lambda)T(\psi,\lambda)R(\lambda)d\lambda
\end{equation}

Let $\vm{d}_{ij}$ be the vector from $i^{th}$ receiving element to $j^{th}$ transmitting element. The distance between the two is then given by $||\vm{d}_{ij}||$. The channel gain from the $i^{th}$ receiving element to the $j^{th}$ transmitting element is given by
\begin{equation}
	\label{eqChGain}
	h_{ij} = L_{m_j}(\phi_{ij})\frac{A_i}{||\vm{d}_{ij}||^{2}}cos(\psi_{ij})R_{e_i}(\psi_{ij})
\end{equation}
where $m_j$ is the Lambertian order of the $j^{th}$ transmitting element, $\phi_{ij}$ and $\psi_{ij}$ are the angles subtended between vector $\vm{d}_{ij}$ and surface normals respectively of the $j^{th}$ transmitting and $i^{th}$ receiving elements, and $A_i$ is the active area of the photodiode on the $i^{th}$ receiving element. The $h_{ij}$ thus computed are the elements of the channel matrix $\vm{H}$.

Ambient light incident on a photodiode generates shot noise. Let $P_a(\lambda)$ be SPD of isotropic ambient light. This would generate shot noise in a receiving element with variance
\begin{equation}
	\label{eqNshot}
	\sigma_{sh}^{2} = \frac{2qAG_{\eta}(\psi)}{\Psi_{\frac{1}{2}}}\int_{\lambda_{min}}^{\lambda_{max}}\int_{0}^{\Psi_{\frac{1}{2}}}P_{a}(\lambda)T(\psi,\lambda)R(\lambda)d\psi d\lambda
\end{equation}
where $q$ is the charge of an electron. Shot noise statistics are typically modeled as a Poisson distribution.

A trans-impedance amplifier (TIA) is most often used as the first stage amplifier. Thermal noise is the most dominant component of TIA electrical noise. The thermal noise variance as described in reference \cite{kah97a} is given by 
\begin{equation}
	\label{eqNth}
	\sigma_{th}^{2} = \frac{4kT}{R_{f}}
\end{equation}
where constant $k$ is Boltzmann constant, $T$ is the absolute temperature, and $R_{f}$ is the TIA feedback resistance. Thermal noise statistics are typically modeled as a Gaussian distribution.

Typically, shot noise due to ambient light and thermal noise from the TIA are dominant over shot noise due to signal \cite{kah97a}. Thus, total input referred noise variance is usually approximated as the sum of shot noise variance and thermal noise variance.  For sake of mathematical simplicity, total noise is assumed to have Gaussian statistics with mean $0$ and variance $\sigma_{n}^{2} = \sigma_{sh}^{2} + \sigma_{th}^{2}$.

In VLC, transmitting elements perform dual function of providing wireless data transmission while maintaining illumination levels. Note that the transmitted radiant flux to achieve a constant target illumination at receiver will vary across different CCTs. Thus, to perform a fair comparison between different modulation schemes transmitting different radiant flux but operating at same illumination levels, needs a different definition of signal to noise ratio. In this work, signal to noise ratio is defined similar to that in reference \cite{fat13a} as the ratio of the \textit{average transmitted} electrical power to noise power. 
\begin{equation}
	\label{eqSNR}
	\text{SNR}^{\text{tx}}_{\text{avg}} := \frac{(eP_{avg}^{tx})^2}{\sigma_n^{2}}
\end{equation}
where $P_{avg}^{tx}$ is the average radiant flux emitted by a transmitter, $e$ is the optical to electrical conversion factor $(AW^{-1}\Omega^{-2})$ and $\sigma_n^{2}$ is the noise power. Parameter $e$ is assumed to be $1$ without loss of generality.