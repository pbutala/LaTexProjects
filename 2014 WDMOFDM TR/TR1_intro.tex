\section{Introduction}
%Spectrum crunch. Solid state lighting wave. VLC as a means to mitigate downlink bottleneck.\\
%Smart spaces.  Multi-colored LEDs. Color tunability + WDM.\\
%ACO-OFDM. DCO-OFDM. O-OFDM background.\\
%In this work, find optimal range of operation for communication.
%O-OFDM over each color WDM + O-OFDM. More capacity.\\
There has been a rapid increase in use of networked portable computing devices in recent years. These devices are consuming increasingly more information in the form of multimedia streaming \cite{cis14a}. Trying to keep up with this increasing demand for wireless data has strained the network infrastructure thus creating the phenomenon of ``spectrum crunch''. Its effect can be seen in reduced quality of service and lower download speeds. 

On the other hand, advances made by the solid-state industry has created energy efficient illumination devices called light emitting diodes (LED). The intensity of radiant flux emitted by LEDs can be modulated at a high enough rate such that information transfer can be achieved at relatively high speeds while the intensity variations are invisible to human eye. One or more LEDs can be packaged together to form a ``luminaire'' which under the above model services the dual functionality of providing wireless network access and maintaining illumination \cite{kom04a}. The additional downlink capacity provided by such ``smart'' luminaires can help mitigate some of the aforementioned spectrum crunch.

A simple visible light communication (VLC) link can be created by using an LED as a transmitter and a photodiode as a receiver. Channel characteristics for an optical channel are described in reference \cite{kah97a}. Multiple transmitting and receiving elements in a multiple input multiple output (MIMO) configuration help increase the capacity of the channel. Different types of MIMO systems \cite{hra06a,zen09a,ash10a,but13a,but14b} have been reported in literature.

One type of an optical MIMO system is wavelength division multiplexed (WDM) VLC system. Different WDM system prototypes have been reported in literature \cite{wan11a,kot12a,cos12a}. These describe an instance of a WDM system without analysis of the optimal operating point. In this work, design of multi-wavelength VLC systems under lighting constraints when correlated color temperature (CCT) of illumination, transmitter spectral power distribution (SPD) and receivers' filter spectral transmittance full width at half maximum (FWHM) are varied, is studied for the first time. Simulations for a three colored WDM VLC system then provide numerical analysis of the system performance giving an insight into optimal design criteria. For the system considered, we find that its most power efficient operation occurs at CCT of 6250 K, narrow transmitting elements' SPD (5 nm), and receiver filter FWHM of 40 nm.

%The following notations are used in this paper. Scalar values are represented in regular font. Vectors and matrices are represented in bold font. Conjugate transpose of $\vm{A}$ is represented by $\vm{A}^{*}$. Operators $:=$ and $||.||$ represent definition and euclidean norm, respectively. 

An introduction to optical MIMO systems is provided in Section \ref{sec:mimo}. WDM, a subset of optical MIMO systems, is introduced in Section \ref{sec:wdm}. Section \ref{sec:simulation} describes the simulation setup. Results and discussion is provided in Section \ref{sec:results}. Conclusions are then drawn in Section \ref{sec:conclusion}. 