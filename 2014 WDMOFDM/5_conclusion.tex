\section{Conclusion}\label{sec:conclusion}
In this work, multi-wavelength VLC system in the context of variable illumination constraints was introduced. VLC system performance was characterized for variations in illumination CCT, transmitter SPD spread, and the receiver filter transmittance FWHM. For the three colored system considered, the blue and the green links pose the performance bottlenecks because of the relatively lower contribution to the SPD and lower photodiode responsivity as compared to the red. As the ICI increases, the most power efficient CCT shifts towards lower temperatures. Transmitting elements with the smallest spectral spread provide the most power efficient operating point. The effect of increase in spectral spread is most pronounced in the green link because it suffers the most from interference from the blue and red links. Filters with narrow transmittance FWHM reject a lot of the signal power while filters with a broad transmittance FWHM accept a lot of interference. Both of these affect the power efficiency of the system. For the setup considered, least power efficient operating point is for DCO-OFDM at CCT = 2500 K, transmitting element SPD spread = 50 nm, and filter FWHM = 1 nm. The most power efficient operating point is for ACO-OFDM at CCT = 6250 K, transmitting element SPD spread = 5 nm, and filter FWHM = 40 nm.