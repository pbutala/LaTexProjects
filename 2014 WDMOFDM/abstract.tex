Light emitting diodes have started replacing the conventional light sources such as incandescent or fluorescent bulbs to provide illumination in indoor spaces. Recent research efforts have tried to smartly deploy these devices to create an optical wireless communication downlink while providing illumination. Use of multi-colored light emitting diodes in a light source provides the ability to (a) tune the color of the light, and (b) establish multiple, parallel wavelength division multiplexed channels. Optical orthogonal frequency division multiplexing provides high spectral efficiency over an optical channel. In this work, we analyze how the performance of wavelength and orthogonal frequency division multiplexing is affected by variations in (a) spectral width of the light emitted from the source, (b) color temperature of aggregate spectrum generated by emitted light, and (c) optical filter bandwidth used for wavelength separation at the receivers. 