Visible light communications (VLC) are achieved by modulation of one or more spectral components in the visible spectrum ($\approx$400-800 nm). The use of this range provides an opportunity to exploit an otherwise untapped medium that is used in human lighting. Most visible light communication systems constructed to date focus on using a broad visible band generated by phosphor-converted blue light emitting diodes, or by filtering to isolate the blue components from these sources. Multi-wavelength systems consider multiple wavelength bands that are combined to produce the desired spectrum realizing a desired color temperature and intensity. The use of multiple bands is also a form of wavelength-division multiplexing. In this paper, we investigate the relationships between the colors comprising the lighting source for a range of lighting states, the spectral separation of communication channels, the relative intensities required to realize lighting states, how modulation can be most effectively mapped to the available color channels, and the design of an optical filtering approach to maximize SNR while minimizing crosstalk at the receiver. Simulation results based on a three colored VLC system are discussed using orthogonal frequency division multiplexing for each color. We show that the system is the most power efficient at 6250 K correlated color temperature, with transmitter spectral spread of 5 nm and filter transmittance width of 40 nm.