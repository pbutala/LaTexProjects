% This file contains all the necessary setup and commands to create
% the preliminary pages according to the buthesis.sty option.

\title{Optical MIMO communication systems under illumination constraints}

\author{Pankil M. Butala}

% Type of document prepared for this degree:
%   1 = Master of Science thesis,
%   2 = Doctor of Philisophy dissertation.
%   3 = Master of Science thesis and Doctor of Philisophy dissertation.
\degree=2

\prevdegrees{B.E., University of Mumbai, 2006\\
	M.S., University of California, Los Angeles, 2007}

\department{Department of Electrical and Computer Engineering}

% Degree year is the year the diploma is expected, and defense year is
% the year the dissertation is written up and defended. Often, these
% will be the same, except for January graduation, when your defense
% will be in the fall of year X, and your graduation will be in
% January of year X+1
\defenseyear{2015}
\degreeyear{2015}

% For each reader, specify appropriate label {First, Second, Third},
% then name, and title. IMPORTANT: The title should be:
%   "Professor of Electrical and Computer Engineering",
% or similar, but it MUST NOT be:
%   Professor, Department of Electrical and Computer Engineering"
% or you will be asked to reprint and get new signatures.
% Warning: If you have more than five readers you are out of luck,
% because it will overflow to a new page. You may try to put part of
% the title in with the name.
\reader{First}{Thomas D.C. Little, Ph.D.}{Professor of Electrical and Computer Engineering\\Professor of Systems Engineering}
\reader{Second}{Jeffrey Carruthers, Ph.D.}{Associate Professor of Electrical and Computer Engineering\\}
\reader{Third}{Bobak Nazer, Ph.D.}{Assistant Professor of Electrical and Computer Engineering\\Assistant Professor of Systems Engineering}
\reader{Fourth}{Valencia Joyner Koomson, Ph.D.}{Associate Professor of Electrical and Computer Engineering\\Tufts University}

% The Major Professor is the same as the first reader, but must be
% specified again for the abstract page. Up to 4 Major Professors
% (advisors) can be defined. 
\numadvisors=1
\majorprof{Thomas D.C. Little, Ph.D.}{{Professor of Electrical and Computer Engineering\\Professor of Systems Engineering}}
%\majorprofb{First M. Last, PhD}{{Professor of computer Science}}
%\majorprofc{First M. Last, PhD}{{Professor of Astronomy}}
%\majorprofd{First M. Last, PhD}{{Professor of Biomedical Engineering}}

%%%%%%%%%%%%%%%%%%%%%%%%%%%%%%%%%%%%%%%%%%%%%%%%%%%%%%%%%%%%%%%%  

%                       PRELIMINARY PAGES
% According to the BU guide the preliminary pages consist of:
% title, copyright (optional), approval,  acknowledgments (opt.),
% abstract, preface (opt.), Table of contents, List of tables (if
% any), List of illustrations (if any). The \tableofcontents,
% \listoffigures, and \listoftables commands can be used in the
% appropriate places. For other things like preface, do it manually
% with something like \newpage\section*{Preface}.

% This is an additional page to print a boxed-in title, author name and
% degree statement so that they are visible through the opening in BU
% covers used for reports. This makes a nicely bound copy. Uncomment only
% if you are printing a hardcopy for such covers. Leave commented out
% when producing PDF for library submission.
%\buecethesistitleboxpage

% Make the titlepage based on the above information.  If you need
% something special and can't use the standard form, you can specify
% the exact text of the titlepage yourself.  Put it in a titlepage
% environment and leave blank lines where you want vertical space.
% The spaces will be adjusted to fill the entire page.
\maketitle
\cleardoublepage

% The copyright page is blank except for the notice at the bottom. You
% must provide your name in capitals.
\copyrightpage
\cleardoublepage

% Now include the approval page based on the readers information
\approvalpage
\cleardoublepage

% Here goes your favorite quote. This page is optional.
\newpage
%\thispagestyle{empty}
\phantom{.}
\vspace{0.7in}

\begin{singlespace}
\begin{quote}
``$\Delta$v - v for velocity, $\Delta$ for change. In space, this is the measure of change in velocity required to get from one place to another, thus a measure of the energy required to do it. Everything is moving already but to get something from the moving surface of the Earth into orbit around it requires a minimum $\Delta$v of 10 km/s. To leave Earth's orbit and fly to Mars requires a minimum $\Delta$v of 3.6 km/s and to orbit Mars and land on it requires a $\Delta$v of about 1 km/s. The hardest part is leaving Earth behind, for that is by far the deepest gravity well involved.''

Kim Stanley Robinson\\
Red Mars (2.2.99) 
%{\color{red} check reference}
\end{quote}
\end{singlespace}

% \vspace{0.7in}
%
% \noindent
% [The descent to Avernus is easy; the gate of Pluto stands open night
% and day; but to retrace one's steps and return to the upper air, that
% is the toil, that the difficulty.]

%\phantom{.}
\vspace{0.7in}

\begin{singlespace}
\begin{quote}
``$\Delta$v - v for velocity, $\Delta$ for change. In space, this is the measure of change in velocity required to get from one place to another, thus a measure of the energy required to do it. Everything is moving already but to get something from the moving surface of the Earth into orbit around it requires a minimum $\Delta$v of 10 km/s. To leave Earth's orbit and fly to Mars requires a minimum $\Delta$v of 3.6 km/s and to orbit Mars and land on it requires a $\Delta$v of about 1 km/s. The hardest part is leaving Earth behind, for that is by far the deepest gravity well involved.''

Kim Stanley Robinson\\
Red Mars (2.2.99) 
%{\color{red} check reference}
\end{quote}
\end{singlespace}

% \vspace{0.7in}
%
% \noindent
% [The descent to Avernus is easy; the gate of Pluto stands open night
% and day; but to retrace one's steps and return to the upper air, that
% is the toil, that the difficulty.]

\cleardoublepage

% The acknowledgment page should go here. Use something like
% \newpage\section*{Acknowledgments} followed by your text.
\newpage
\section*{\centerline{Acknowledgments}}
%Here go all your acknowledgments. You know, your advisor, funding agency, lab
%mates, etc., and of course your family.
%

%\hspace{2.75ex} To my parents, Nita and Mukund Butala - 
%%No amount of gratitude can ever express how much I am indebted to you two. I would have accomplished naught without your numerous sacrifices, unconditional love and support. I am proud to be your son.
%
%To my grandmother, Hasumati - 
%%Ba, you 
%
%To my advisor, Prof. Little - 
%
%To my co-advisor, Dr. Hany Elgala -
%
%To my committee members, Prof. Carruthers, Prof. Nazer and Prof. Koomson - 
%
%To my lab mates, Mike, Jimmy, Yuting - 
%
%To NSF - 
%To my uncle Rajiv, aunt Sangita and sisters Sneha and Shyla - 
%
%Finally, dear Anuja - 
%
%\vskip 1in
%\noindent
%Sincerely,\\
%Pankil.

I would like to thank ...
\vskip 1in

\noindent
Sincerely,\\
Pankil
%%Here go all your acknowledgments. You know, your advisor, funding agency, lab
%mates, etc., and of course your family.
%

%\hspace{2.75ex} To my parents, Nita and Mukund Butala - 
%%No amount of gratitude can ever express how much I am indebted to you two. I would have accomplished naught without your numerous sacrifices, unconditional love and support. I am proud to be your son.
%
%To my grandmother, Hasumati - 
%%Ba, you 
%
%To my advisor, Prof. Little - 
%
%To my co-advisor, Dr. Hany Elgala -
%
%To my committee members, Prof. Carruthers, Prof. Nazer and Prof. Koomson - 
%
%To my lab mates, Mike, Jimmy, Yuting - 
%
%To NSF - 
%To my uncle Rajiv, aunt Sangita and sisters Sneha and Shyla - 
%
%Finally, dear Anuja - 
%
%\vskip 1in
%\noindent
%Sincerely,\\
%Pankil.

I would like to thank ...
\vskip 1in

\noindent
Sincerely,\\
Pankil
\cleardoublepage

% The abstractpage environment sets up everything on the page except
% the text itself.  The title and other header material are put at the
% top of the page, and the supervisors are listed at the bottom.  A
% new page is begun both before and after.  Of course, an abstract may
% be more than one page itself.  If you need more control over the
% format of the page, you can use the abstract environment, which puts
% the word "Abstract" at the beginning and single spaces its text.

\begin{abstractpage}
% ABSTRACT
Technology for wireless information access has enabled innovation of `smart' portable consumer devices. These have been widely adopted and have become an integral part of our daily lives. These along with internet-of-things need ubiquitous connectivity to the internet to provide value added services, maximize their functionality and create a smarter world to live in. Cisco's visual networking index predicts wireless data consumption to increase by a cumulative rate of 61\% every year. This will put additional stress on the already stressed wireless access network infrastructure creating a phenomenon called `spectrum crunch'. 

On the other hand, solid state devices industry has made remarkable advances in energy efficient light-emitting-diodes (LED). The lighting industry is rapidly adopting LEDs to provide illumination in indoor spaces. Lighting fixtures are positioned to support human activities and thus are well located to act as wireless access points. The visible spectrum (380 nm -- 780 nm) is yet unregulated and untapped for wireless access. This provides unique opportunity to upgrade existing lighting infrastructure and create a dense grid of small cells by using this additional `optical' wireless bandwidth. Under the above model, lighting fixtures will service dual missions of illumination and wireless access points. 

This dissertation investigates multiple-input multiple-output (MIMO) optical wireless broadcast system under unique constraints imposed by the optical channel and illumination requirements. Sample indexed spatial orthogonal frequency division multiplexing and metameric modulation are proposed to achieve higher spectral efficiency by exploiting dimensions of space and color respectively in addition to time and frequency. A framework is developed to analyze performance of optical MIMO imaging systems. Performance improvements for optical systems have been achieved by decorrelating spatially separate links by incorporating an imaging receiver. The dissertation also studies the impact of visual perception on performance color shift keying as specified in IEEE 802.15.7 standard. Luminous-signal-to-noise ratio metric is introduced to compare performance of optical modulation techniques operating at same illumination intensity. The dissertation then introduces singular value decomposition based OWC system architecture to incorporate illumination constraints independent of communication constraints in a MIMO system. It then studies design paradigm for a multi--colored wavelength division multiplexed indoor OWC system. 
%% ABSTRACT
Technology for wireless information access has enabled innovation of `smart' portable consumer devices. These have been widely adopted and have become an integral part of our daily lives. These along with internet-of-things need ubiquitous connectivity to the internet to provide value added services, maximize their functionality and create a smarter world to live in. Cisco's visual networking index predicts wireless data consumption to increase by a cumulative rate of 61\% every year. This will put additional stress on the already stressed wireless access network infrastructure creating a phenomenon called `spectrum crunch'. 

On the other hand, solid state devices industry has made remarkable advances in energy efficient light-emitting-diodes (LED). The lighting industry is rapidly adopting LEDs to provide illumination in indoor spaces. Lighting fixtures are positioned to support human activities and thus are well located to act as wireless access points. The visible spectrum (380 nm -- 780 nm) is yet unregulated and untapped for wireless access. This provides unique opportunity to upgrade existing lighting infrastructure and create a dense grid of small cells by using this additional `optical' wireless bandwidth. Under the above model, lighting fixtures will service dual missions of illumination and wireless access points. 

This dissertation investigates multiple-input multiple-output (MIMO) optical wireless broadcast system under unique constraints imposed by the optical channel and illumination requirements. Sample indexed spatial orthogonal frequency division multiplexing and metameric modulation are proposed to achieve higher spectral efficiency by exploiting dimensions of space and color respectively in addition to time and frequency. A framework is developed to analyze performance of optical MIMO imaging systems. Performance improvements for optical systems have been achieved by decorrelating spatially separate links by incorporating an imaging receiver. The dissertation also studies the impact of visual perception on performance color shift keying as specified in IEEE 802.15.7 standard. Luminous-signal-to-noise ratio metric is introduced to compare performance of optical modulation techniques operating at same illumination intensity. The dissertation then introduces singular value decomposition based OWC system architecture to incorporate illumination constraints independent of communication constraints in a MIMO system. It then studies design paradigm for a multi--colored wavelength division multiplexed indoor OWC system. 
\end{abstractpage}
\cleardoublepage

% Now you can include a preface. Again, use something like
% \newpage\section*{Preface} followed by your text

% Table of contents comes after preface
\tableofcontents
\cleardoublepage

% If you do not have tables, comment out the following lines
\newpage
\listoftables
\cleardoublepage

% If you have figures, uncomment the following line
\newpage
\listoffigures
\cleardoublepage

% List of Abbrevs is NOT optional (Martha Wellman likes all abbrevs listed)
\chapter*{List of Abbreviations}
\begin{center}
  \begin{longtable}{lll}
    \hspace*{2em} & \hspace*{1in} & \hspace*{4.5in} \\
    ACO & \dotfill & Asymmetrically Clipped Optical \\
		APD & \dotfill & Avalanche Photo Diode \\
		AWGN & \dotfill & Additive White Gaussian Noise \\
		CIE & \dotfill & Commission Internationale de l'Eclairage \\
		CSI & \dotfill & Channel State Information \\
		CSK & \dotfill & Color Shift Keying \\
		DCO & \dotfill & DC biased Optical \\
		DMT & \dotfill & Discrete Multi-Tone \\
		FOV & \dotfill & Field Of View \\
		GCS & \dotfill & Global Coordinate System \\
		ICI & \dotfill & Inter Channel Interference \\
		IID & \dotfill & Independent and Identically Distributed \\
		IM/DD & \dotfill & Intensity Modulation / Direct Detection \\
		LED & \dotfill & Light Emitting Diode \\
		LOS & \dotfill & Line Of Sight \\
		MIMO & \dotfill & Multiple Input Multiple Output \\
		MM & \dotfill & Metameric Modulation \\
		OFDM  & \dotfill & Orthogonal Frequency Division Multiplexing \\
		OOK & \dotfill & On-Off Keying \\
		PAM & \dotfill & Pulse Amplitude Modulation \\
		PCS & \dotfill & Primary Color Space \\
		PD & \dotfill & Photo Diode \\
		PIN & \dotfill & P-I-N Junction \\
		PPM & \dotfill & Pulse Position Modulation \\
		PWM & \dotfill & Pulse Width Modulation \\
		QAM & \dotfill & Quadrature Amplitude Modulation \\
		RCS & \dotfill & Receiver Coordinate System \\
    RF  & \dotfill & Radio Frequency \\
		RGB & \dotfill & Red, Green and Blue \\
		SISO & \dotfill & Single Input Single Output \\
		SIS & \dotfill & Sample Indexed Spatial \\
		SM & \dotfill & Spatial Modulation \\
		SMP & \dotfill & Spatial Multiplexing \\
		SNR & \dotfill & Signal to Noise Ratio \\
		SPD & \dotfill & Spectral Power Distribution \\
		SSK & \dotfill & Spatial Shift Keying \\
		SVD & \dotfill & Singular Value Decomposition \\
		TIA & \dotfill & Trans-Impedance Amplifier \\
		UCS & \dotfill & Universal Color Space \\
		VCS & \dotfill & Visual Color Space \\
    VLC  & \dotfill & Visible Light Communication \\
		VPPM & \dotfill & Variable Pulse Position Modulation \\
		WDM & \dotfill & Wavelength Division Multiplexing \\
		& \dotfill & \\
    $\mathbb{R}^{2}$  & \dotfill & the Real plane \\
  \end{longtable}
\end{center}
%\chapter*{List of Abbreviations}
\begin{center}
  \begin{longtable}{lll}
    \hspace*{2em} & \hspace*{1in} & \hspace*{4.5in} \\
    ACO & \dotfill & Asymmetrically Clipped Optical \\
		APD & \dotfill & Avalanche Photo Diode \\
		AWGN & \dotfill & Additive White Gaussian Noise \\
		CIE & \dotfill & Commission Internationale de l'Eclairage \\
		CSI & \dotfill & Channel State Information \\
		CSK & \dotfill & Color Shift Keying \\
		DCO & \dotfill & DC biased Optical \\
		DMT & \dotfill & Discrete Multi-Tone \\
		FOV & \dotfill & Field Of View \\
		GCS & \dotfill & Global Coordinate System \\
		ICI & \dotfill & Inter Channel Interference \\
		IID & \dotfill & Independent and Identically Distributed \\
		IM/DD & \dotfill & Intensity Modulation / Direct Detection \\
		LED & \dotfill & Light Emitting Diode \\
		LOS & \dotfill & Line Of Sight \\
		MIMO & \dotfill & Multiple Input Multiple Output \\
		MM & \dotfill & Metameric Modulation \\
		OFDM  & \dotfill & Orthogonal Frequency Division Multiplexing \\
		OOK & \dotfill & On-Off Keying \\
		PAM & \dotfill & Pulse Amplitude Modulation \\
		PCS & \dotfill & Primary Color Space \\
		PD & \dotfill & Photo Diode \\
		PIN & \dotfill & P-I-N Junction \\
		PPM & \dotfill & Pulse Position Modulation \\
		PWM & \dotfill & Pulse Width Modulation \\
		QAM & \dotfill & Quadrature Amplitude Modulation \\
		RCS & \dotfill & Receiver Coordinate System \\
    RF  & \dotfill & Radio Frequency \\
		RGB & \dotfill & Red, Green and Blue \\
		SISO & \dotfill & Single Input Single Output \\
		SIS & \dotfill & Sample Indexed Spatial \\
		SM & \dotfill & Spatial Modulation \\
		SMP & \dotfill & Spatial Multiplexing \\
		SNR & \dotfill & Signal to Noise Ratio \\
		SPD & \dotfill & Spectral Power Distribution \\
		SSK & \dotfill & Spatial Shift Keying \\
		SVD & \dotfill & Singular Value Decomposition \\
		TIA & \dotfill & Trans-Impedance Amplifier \\
		UCS & \dotfill & Universal Color Space \\
		VCS & \dotfill & Visual Color Space \\
    VLC  & \dotfill & Visible Light Communication \\
		VPPM & \dotfill & Variable Pulse Position Modulation \\
		WDM & \dotfill & Wavelength Division Multiplexing \\
		& \dotfill & \\
    $\mathbb{R}^{2}$  & \dotfill & the Real plane \\
  \end{longtable}
\end{center}
\cleardoublepage

% END OF THE PRELIMINARY PAGES

\newpage
\endofprelim
