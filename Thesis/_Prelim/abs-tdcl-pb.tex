% ABSTRACT
% define OWC - done
% Try to remove items that will date the work (E.g., IoT) - done
% Best to state some results or conclusions here (you are stating what you did not, the results) - done

Technology for wireless information access has enabled innovation of `smart' portable consumer devices. These have been widely adopted and have become an integral part of our daily lives. They need ubiquitous connectivity to the internet to provide value added services, maximize their functionality and create a smarter world to live in. Cisco's visual networking index currently predicts wireless data consumption to increase by 61\% per year. This will put additional stress on the already stressed wireless access network infrastructure creating a phenomenon called `spectrum crunch'.

At the same time, the solid state devices industry has made remarkable advances in energy efficient light-emitting-diodes (LED). The lighting industry is rapidly adopting LEDs to provide illumination in indoor spaces. Lighting fixtures are positioned to support human activities and thus are well located to act as wireless access points. The visible spectrum (380 nm -- 780 nm) is yet unregulated and untapped for wireless access. This provides unique opportunity to upgrade existing lighting infrastructure and create a dense grid of small cells by using this additional `optical' wireless bandwidth. Under the above model, lighting fixtures will service dual missions of illumination and access points for optical wireless communication (OWC).

This dissertation investigates multiple-input multiple-output (MIMO) optical wireless broadcast system under unique constraints imposed by the optical channel and illumination requirements. Sample indexed spatial orthogonal frequency division multiplexing (SIS-OFDM) and metameric modulation (MM) are proposed to achieve higher spectral efficiency by exploiting dimensions of space and color respectively in addition to time and frequency. SIS-OFDM can provide significant additional spectral efficiency of up to ($N_{\text{sc}}$/2 - 1) $\times$ $k$ bits/symbol where $N_{\text{sc}}$ is total number of subcarriers and $k$ is number of bits per underlying spatial modulation symbol. MM always generates the true requested illumination color and has the potential to provide better color rendering by incorporating multiple LEDs. A normalization framework is then developed to analyze performance of optical MIMO imaging systems. Performance improvements of up to 45 dB for optical systems have been achieved by decorrelating spatially separate links by incorporating an imaging receiver. The dissertation also studies the impact of visual perception on performance color shift keying as specified in IEEE 802.15.7 standard. It shows that non-linearity for a practical system can have a performance penalty of up to 15 dB when compared to the simplified linear system abstraction as proposed in the standard. Luminous-signal-to-noise ratio, a novel metric is introduced to compare performance of optical modulation techniques operating at same illumination intensity. The dissertation then introduces singular value decomposition based OWC system architecture to incorporate illumination constraints independent of communication constraints in a MIMO system. It then studies design paradigm for a multi--colored wavelength division multiplexed indoor OWC system.
