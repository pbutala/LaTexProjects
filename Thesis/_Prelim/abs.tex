% ABSTRACT
In recent years, there has been a large-scale adoption of portable computing devices like smartphones and tablets. These along with internet-of-things need ubiquitous connectivity to the internet to provide value added services, maximize their functionality and create a `smart'-er world to live in. Cisco's visual networking index predicts wireless data consumption to increase by a cumulative rate of 61\% every year. This will put additional stress on our already stressed wireless access network infrastructure creating a phenomenon called `spectrum crunch'. Wireless access technologies in and around the 60 GHz spectrum along with carrier aggregation adopted by long term evolution (LTE) standards promise to mitigate the spectrum crunch to an extent.

On the other hand, solid state devices industry has made remarkable advances in energy efficient light-emitting-diodes (LED). The lighting industry is rapidly adopting LEDs to provide illumination in indoor spaces. If light emitted from lighting fixtures can be exploited to carry wireless data, these fixtures are then uniquely positioned to act as wireless access points. The visible spectrum (380 nm -- 780 nm) is yet unregulated and untapped for wireless access. Unlike with omni-directional radio frequency (RF) access points, directionality of light can enable creation of a grid of multiple optical access points in close proximity to each other. This provides unique opportunity to upgrade existing lighting infrastructure and create a dense grid of small cells by using this additional `optical' wireless bandwidth. Under the above model, lighting fixtures will service dual missions of illumination and wireless data access. These can be easily set up to function within a heterogeneous system with an optical downlink and an RF uplink. 

This dissertation explores different modulation techniques for an optical wireless broadcast system using intensity modulation / direct detection (IM/DD) under unique constraints imposed by IM/DD and user illumination requirements. Performance of single--input single--output (SISO) optical modulation techniques between a single optical source and a single receiver is studied. The dissertation then explores degrees of freedom provided by frequency, space and color to study performance of multiple--input multiple--output (MIMO) optical modulation techniques between multiple optical sources and multiple receiving elements. This dissertation then introduces several optical MIMO modulation schemes for spectrally efficient MIMO optical wireless communications.