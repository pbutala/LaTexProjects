% ABSTRACT
% a few edits done
% define OWC
% Try to remove items that will date the work (E.g., IoT)
% Best to state some results or conclusions here (you are stating what you did not, the results)

Technology for wireless information access has enabled innovation of
`smart' portable consumer devices. These have been widely adopted and
have become an integral part of our daily lives. These along with
internet-of-things need ubiquitous connectivity to the internet to
provide value added services, maximize their functionality and create
a smarter world to live in. Cisco's visual networking index currenlty
predicts wireless data consumption to increase by 61\% per year. This
will put additional stress on the already stressed wireless access
network infrastructure creating a phenomenon called `spectrum crunch'.

At the same time, the solid state devices industry has made remarkable
advances in energy efficient light-emitting-diodes (LED). The lighting
industry is rapidly adopting LEDs to provide illumination in indoor
spaces. Lighting fixtures are positioned to support human activities
and thus are well located to act as wireless access points. The
visible spectrum (380 nm -- 780 nm) is yet unregulated and untapped
for wireless access. This provides unique opportunity to upgrade
existing lighting infrastructure and create a dense grid of small
cells by using this additional `optical' wireless bandwidth. Under the
above model, lighting fixtures will service dual missions of
illumination and wireless access points.

This dissertation investigates multiple-input multiple-output (MIMO)
optical wireless broadcast system under unique constraints imposed by
the optical channel and illumination requirements. Sample indexed
spatial orthogonal frequency division multiplexing and metameric
modulation are proposed to achieve higher spectral efficiency by
exploiting dimensions of space and color respectively in addition to
time and frequency. A framework is developed to analyze performance of
optical MIMO imaging systems. Performance improvements for optical
systems have been achieved by decorrelating spatially separate links
by incorporating an imaging receiver. The dissertation also studies
the impact of visual perception on performance color shift keying as
specified in IEEE 802.15.7 standard. Luminous-signal-to-noise ratio
metric is introduced to compare performance of optical modulation
techniques operating at same illumination intensity. The dissertation
then introduces singular value decomposition based OWC system
architecture to incorporate illumination constraints independent of
communication constraints in a MIMO system. It then studies design
paradigm for a multi--colored wavelength division multiplexed indoor
OWC system.
