%\section{Summary of the dissertation}
%\label{sec:summary}
%\graphicspath{{_Conclusion/Figures/}}
Technology for wireless information access has enabled innovation of `smart' portable consumer devices. These have been widely adopted and have become an integral part of our daily lives. Networking forecasts indicate a rapid rise in data consumption by portable devices in years to come. Wireless access infrastructure in its current state cannot keep pace with this rise in demand for wireless network access. Thus there is a need to create additional wireless bandwidth by not only improving utilization of current infrastructure but also exploiting additional spectrum.

Advances made in construction and manufacturing of solid state devices have created energy efficient LEDs. Lighting industry is rapidly adopting these devices for providing illumination in indoor spaces. Lighting fixtures are typically located to assist human activities and are thus uniquely positioned to act as access points. These trends have created a unique opportunity to utilize the unregulated optical spectrum and provide additional wireless network access. 

This dissertation investigates MIMO OWC systems under illumination targets and improves upon the state of the art by better utilizing the spatial and color dimensions. Since the primary function of the luminaires is to provide illumination, incorporation of illumination targets while seamlessly providing wireless access is important for such systems to be practically adopted. Additionally indoor OWC systems must provide data rates equal to or greater than those provided by existing RF wireless technology to effectively mitigate the wireless bottleneck. Thus it is important to improve the performance of optical modulation techniques and achievable spectral efficiencies by exploiting spatial and color dimensions.

Imaging receivers have been shown to reduce effect of shot noise on the optical communication channel \cite{dja00a}. To analyze performance of OWC system with an imaging receiver, a novel receiver normalization framework is developed \cite{but14a}. It is shown that incorporating an imaging receiver can improve performance of SM and SMP by up to 45 dB by effectively decorrelating the spatial streams. It is also shown that for an imaging OWC system at lower spectral efficiencies, SM performs better than SMP where as at higher spectral efficiencies, SMP performs better than SM.

SIS-OFDM, a spectrally efficient modulation technique has been proposed \cite{but14b}. It improves upon the performance of modulation technique described in \cite{zha12a} by carrying out SM on O-OFDM time domain samples rather than frequency domain subcarriers. This technique allows transmission of additional $3\times N_{\text{sc}}\times k/4$ bits/symbol with ACO-OFDM and $(N_{\text{sc}}/2 - 1)\times k$ bits/symbol with DCO-OFDM where $N_{\text{sc}}$ is total number of subcarriers and $k$ is bits per underlying SM symbol.

LSNR, a metric to compare performance of different modulation techniques at the same illumination intensity levels is proposed. It is used to compare performance of CSK and MM.

Performance of CSK is studied under the linear system model as specified in the IEEE 802.15.7 standard \cite{IEEE802.15.7}. Given the inherent non--linearity in human visual perception, a non--linear model for CSK is proposed. It is shown that the non--linearity causes performance penalties of 15 dB, 10 dB and 5 dB over the linear model for 4, 8 and 16-CSK when studied for CBCs specified in the standard. However there is scope for improvement by optimizing the CSK constellation for skewed noise characteristics in the CIE-CS chromaticity plane.

MM has been proposed to provide OWC without data--dependent color flicker that may occur in CSK systems \cite{but12a}. MM always generates the true requested illumination color and has the potential to provide better color rendering. Performance of MM with different CBCs are compared and analyzed. It is shown that 4-MM with 6 LEDs forming CBC sets \{1, 2, 4, 5\}, \{1, 2, 4, 6\}, \{1, 2, 4, 7\} and \{1, 2, 4, 8\} outperform others.

An novel OWC system design using SVD techniques to combine illumination and communication information is described \cite{but13a}. It is shown that it is possible to achieve high data rate communications while maintaining requested illumination profile in indoor spaces. The system requires CSI to be known by the system adding some complexity.

A multi-wavelength OWC system design paradigm is also studied \cite{but14c}. It is shown that at as ICI increases, the most power efficient CCT shifts towards lower temperatures. Transmitting elements with smallest spectral spreads provide efficient communications. Filters with narrow FWHM reject a lot of signal power while those with wide FWHM accept a lot of interference.

As a part of smart lighting engineering research center, various prototypes and proof-of-concept demonstrations have been developed. CuSP, a color sensor platform was developed to support smart--room controls in indoor spaces. A prototype $4\times 4$ MIMO OWC system was also specified in collaboration with university partners. $2\times 2$ MIMO OWC system concept was prototyped and demonstrated.