\section{Related work}
\label{sec:related}
\graphicspath{{_Intro/Figures/}}

For OWC to be a viable candidate for mitigating `bottleneck' on wireless downlink, the achievable data rates per user need to be at least of the same order as those using RF. Wireless data rates are directly proportional to a function of achievable bandwidth and spectral efficiency of modulation techniques. Research to improve the performance of OWC has gained traction in recent years. These have been focused on improving the achievable bandwidth using LEDs and spectral efficiency of employed modulation techniques. Along the way, a few experimental prototypes have also been reported. 

\subsection{Enhancing LED bandwidth}
\label{relatedBandwidth}
At conventional office lighting illumination levels (400 lux), a single-input single-output (SISO) optical communication channel operates at a high signal-to-noise ratio (SNR) but is limited in its capacity due to the inherent low switching speeds of high brightness LEDs used for illumination (1 MHz -- 2 MHz). In a phosphor converted white LED, while the blue LED can be modulated at higher frequencies, the phosphor conversion process is a relatively slow process thus limiting the bandwidth of the LED. \cite{gru08b} used a blue filter at the receiver to extract just the blue part of the incident spectrum and thus improve the modulation 3 dB bandwidth of the LED to about 20 MHz; however at the cost of the received signal power.  In \cite{min08a}, the authors use resonance equalization at the transmitter to achieve an overall 3 dB bandwidth of about 25 MHz. In \cite{zen08a}, the authors use an equalizer at the receiver to extend the channel 3 dB bandwidth to about 50 MHz. In \cite{tso14a}, $\mu$LEDs have been shown to provide 3 dB bandwidth of 60 MHz. However, feasibility of $\mu$LEDs for illumination purposes is yet to be tested. The LED device itself has a slow roll--off in its frequency response beyond its 3 dB attenuation point. This makes it possible to transmit information at higher frequencies using lower order modulation techniques corresponding to lower achievable SNRs while maintaining a target bit error rate (BER).

\subsection{Enhancing spectral efficiency}
\label{relatedSpectral}
Another way to increase the datarate of OWC is to implement spectrally efficient modulation techniques. For this, orthogonal frequency division multiplexing (OFDM) or discrete multi--tone (DMT) modulation have been modified to meet the optical constraints \cite{vuc09a,mes10a,mes10b,dis11a}. In \cite{car96a}, a multiple subcarrier modulation (MSM) technique is implemented by adding DC bias to a real valued time domain OFDM symbol to minimize signal clipping. This technique is also known as DCO-OFDM. While this technique does increase the spectral efficiency, it is power inefficient. In ACO-OFDM \cite{arm06a}, half the spectral efficiency of DCO-OFDM is sacrificed for power efficiency. In this technique, data is assigned to only the odd subcarriers and the time domain OFDM symbol is clipped below zero. Noise introduced due to clipping in this manner has been shown to be orthogonal to the signal. The entire symbol (with additive noise) can be reconstructed at the receiver and transmitted data can be recovered. Higher spectral efficiencies can be achieved by using multiple transmitting elements with multiple receiving elements in a multiple--input multiple--output (MIMO) system. Such a MIMO system can exploit additional dimensions of space and color. Spatial modulation (SM) \cite{mes06a} exploits the spatial dimension by encoding bits in the index of an active transmitter. Color shift keying (CSK) \cite{IEEE802.15.7} exploits the color dimension by encoding bits in the color of the transmitted flux.

\subsection{Experimental prototypes}
\label{relatedExperiments}
In \cite{vuc10a}, the authors experimentally achieve 513 Mb/s data rates over a SISO link for BER $\leq 2\times10^{-3}$. The transmitter implemented DMT with 127 subcarriers within a bandwidth of 100 MHz (beyond the 3dB bandwidth); each modulated with different $M$-ary quadrature amplitude modulation (QAM) constellation based on bit and power loading. An equalizer was implemented at the receiver based on the estimated channel. The system was operated at 1000 lx illumination at the receiver. \cite{tso14a} implement optical OFDM techniques with 512 subcarriers, pre--equalization and bit and power loading along with gallium nitride $\mu$LED front end to demonstrate feasibility of 3 Gb/s SISO link. In \cite{cos12a}, the authors experimentally achieve 1.5 Gb/s for a SISO link and 3.4 Gb/s data rates over a wavelength division multiplexed (WDM) MIMO link for BER $\leq 2*10^{-3}$ using commercial red, green and blue (RGB) LED. In both cases, the transmitter implemented DMT with 512 subcarriers within a bandwidth of 250 MHz; each modulated with different $M$-QAM constellation based on bit and power loading. An equalizer was implemented at the receiver based on the estimated channel. The system was operated at 410 lx illumination at the receiver. For SM, a non-imaging receiver suffers from outages at symmetry points. At the same time the channel matrix is ill-conditioned \cite{zen09a}.  While coverage can be improved using angle diversity receivers \cite{car00a}, the performance can be enhanced remarkably by considering an imaging receiver to decorrelate the coefficients of the MIMO channel matrix \cite{dja00a}. The imaging receiver architecture \cite{kah98a} has the potential to provide the highest capacity for a VLC channel along being incorporated in a handheld device. A spatial multiplexing indoor MIMO technique for VLC technology using OFDM and a imaging receiver is considered in \cite{azh13a}. Here a 4x9 MIMO system is implemented. Here, the four transmitters implement DCO-OFDM with 32 subcarriers within a bandwidth of 4 MHz; each modulated with $M$-QAM symbols. Blue filtering and equalization is implemented at the receiver. The authors experimentally achieve 1 Gb/s transmission with average BER $\leq 10^{-3}$ at an illumination of 1000 lx. Some research has been carried out to achieve low datarate optical communications using cameras in mobile devices. CamCom protocol \cite{rob13a} is being developed by researchers at Intel to integrate OWC with existing portable devices equipped with cameras.



\subsection{IEEE 802.15 task group 7}
\label{relatedStandard}
The IEEE 802.15.7 task group has drafted medium access control (MAC) and physical layer (PHY) standards to support free space optical communications \cite{IEEE802.15.7}. These specifications enable high data rate communications along with dimming. The current LED infrastructure can support the PHY\rmnum{3} specifications \cite{raj12a}. The standard outlines CSK with a linear system model to implement OWC with PHY \rmnum{3}. In CSK, bits are encoded in the color that the luminaire produces. To transmit a bit sequence, the luminaire produces different colors which are averaged out by the human eye to produce the `white' set point. The standard supports up to 16-CSK at 24 MHz clock rate which can provide data rates up to 96 Mb/s.
