\section{Summary of accomplishments}
\label{sec:accomplishments}
\graphicspath{{_Intro/Figures/}}

This dissertation investigates modulation techniques in conjunction with IM/DD for indoor optical wireless broadcast systems in presence of user requested illumination targets. In chapter \ref{chapter:mimoSpace}, a novel framework is proposed to analyze performance of optical imaging MIMO systems. Performance improvements of up to 45 dB for optical systems have been achieved by decorrelating spatially separate links by incorporating an imaging receiver. In chapter \ref{chapter:sisofdm}, we propose sample indexed spatial orthogonal frequency division multiplexing (SIS-OFDM) - a novel MIMO modulation technique that exploits the spatial, temporal and frequency dimensions in order to achieve high spectral efficiency of a MIMO OWC system while maintaining relatively low system complexity. SIS-OFDM can provide significant additional spectral efficiency of up to ($N_{\text{sc}}$/2 - 1) $\times$ $k$ bits/symbol where $N_{\text{sc}}$ is total number of subcarriers and $k$ is number of bits per underlying SM symbol. Human visual perception can be characterized by the \textit{commission internationale de l'eclairage} (CIE) 1931 XYZ color space. It is shown in chapter \ref{chapter:mimoColor} that the non-linearity introduced by visual perception for a practical CSK system can have a performance penalty of up to 15 dB when compared to the simplified linear CSK system abstraction as proposed in the IEEE 802.15.7 standard. Luminous-signal-to-noise ratio (LSNR), a novel metric to compare performance of different signaling techniques operating at same illumination intensity levels, is also introduced. Metameric modulation (MM) is proposed in chapter \ref{chapter:metameric} and is studied as a MIMO signaling technique that exploits the color dimension with multiple sets of LEDs to improve spectral efficiency. We show that MM always generates the true requested illumination color and has the potential to provide better color rendering by incorporating multiple LEDs. Chapter \ref{chapter:system} then proposes a novel technique by using singular value decomposition (SVD) based OWC system architecture to incorporate illumination constraints independent of communication constraints in a MIMO system. It then provides a novel analysis to study performance of a multi--colored wavelength division multiplexed indoor OWC system under different system parameters.

As a part of the smart lighting engineering research center, during this dissertation, prototypes and proof-of-concept demonstrations have been developed in collaboration with partners at Tufts University and Rensselaer Polytechnic Institute. Specifications were generated to develop a 4$\times$4 optical MIMO system with an imaging receiver to operate it at 400 lx illumination level. A color sensor platform (CuSP) has also been developed to create a network of color sensor platforms in a smart space that support smart controls.