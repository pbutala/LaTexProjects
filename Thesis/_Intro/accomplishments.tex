\section{Summary of Accomplishments}
\label{sec:accomplishments}
\graphicspath{{_Intro/Figures/}}

This dissertation investigates modulation techniques in conjunction with IM/DD for indoor optical wireless broadcast systems in presence of user requested illumination targets. A framework is developed to analyze performance of imaging MIMO systems. Performance improvements for optical systems have been achieved by decorrelating spatially separate links by incorporating an imaging receiver. Sample indexed spatial orthogonal frequency division multiplexing (SIS-OFDM) - a novel MIMO modulation technique that exploits the spatial, temporal and frequency dimensions is introduced in order to achieve high spectral efficiency of a MIMO OWC system while maintaining relatively low system complexity. Human visual perception can be characterized by the \textit{commission internationale de l'eclairage} (CIE) 1931 XYZ color space. Impact of non-linearity of this space on performance of CSK is then studied under a non-linear system model. Luminous-signal-to-noise ratio (LSNR), a metric to compare performance of different signaling techniques operating at same illumination intensity levels, is introduced. Metameric modulation (MM) is also introduced and studied as a MIMO signaling technique that exploits the color dimension with multiple sets of LEDs to improve spectral efficiency. The dissertation then introduces the singular value decomposition (SVD) based OWC system architecture to incorporate illumination constraints independent of communication constraints in a MIMO system. It then studies design paradigm for a multi--colored wavelength division multiplexed indoor OWC system. 

As a part of the smart lighting engineering research center, during this dissertation, various prototypes and proof-of-concept demonstrations have been developed in collaboration with partners at Tufts University and Rensselaer Polytechnic Institute. Specifications were generated to develop a 4$\times$4 optical MIMO system with an imaging receiver to operate it at 400 lx illumination level. A color sensor platform (CuSP) has also been developed to create a network of color sensor platforms in a smart space that support smart controls.